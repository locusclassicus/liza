% Options for packages loaded elsewhere
\PassOptionsToPackage{unicode}{hyperref}
\PassOptionsToPackage{hyphens}{url}
%
\documentclass[
]{article}
\usepackage{amsmath,amssymb}
\usepackage{iftex}
\ifPDFTeX
  \usepackage[T1]{fontenc}
  \usepackage[utf8]{inputenc}
  \usepackage{textcomp} % provide euro and other symbols
\else % if luatex or xetex
  \usepackage{unicode-math} % this also loads fontspec
  \defaultfontfeatures{Scale=MatchLowercase}
  \defaultfontfeatures[\rmfamily]{Ligatures=TeX,Scale=1}
\fi
\usepackage{lmodern}
\ifPDFTeX\else
  % xetex/luatex font selection
\fi
% Use upquote if available, for straight quotes in verbatim environments
\IfFileExists{upquote.sty}{\usepackage{upquote}}{}
\IfFileExists{microtype.sty}{% use microtype if available
  \usepackage[]{microtype}
  \UseMicrotypeSet[protrusion]{basicmath} % disable protrusion for tt fonts
}{}
\makeatletter
\@ifundefined{KOMAClassName}{% if non-KOMA class
  \IfFileExists{parskip.sty}{%
    \usepackage{parskip}
  }{% else
    \setlength{\parindent}{0pt}
    \setlength{\parskip}{6pt plus 2pt minus 1pt}}
}{% if KOMA class
  \KOMAoptions{parskip=half}}
\makeatother
\usepackage{xcolor}
\usepackage[margin=1in]{geometry}
\usepackage{graphicx}
\makeatletter
\def\maxwidth{\ifdim\Gin@nat@width>\linewidth\linewidth\else\Gin@nat@width\fi}
\def\maxheight{\ifdim\Gin@nat@height>\textheight\textheight\else\Gin@nat@height\fi}
\makeatother
% Scale images if necessary, so that they will not overflow the page
% margins by default, and it is still possible to overwrite the defaults
% using explicit options in \includegraphics[width, height, ...]{}
\setkeys{Gin}{width=\maxwidth,height=\maxheight,keepaspectratio}
% Set default figure placement to htbp
\makeatletter
\def\fps@figure{htbp}
\makeatother
\setlength{\emergencystretch}{3em} % prevent overfull lines
\providecommand{\tightlist}{%
  \setlength{\itemsep}{0pt}\setlength{\parskip}{0pt}}
\setcounter{secnumdepth}{-\maxdimen} % remove section numbering
\ifLuaTeX
  \usepackage{selnolig}  % disable illegal ligatures
\fi
\IfFileExists{bookmark.sty}{\usepackage{bookmark}}{\usepackage{hyperref}}
\IfFileExists{xurl.sty}{\usepackage{xurl}}{} % add URL line breaks if available
\urlstyle{same}
\hypersetup{
  pdftitle={Liza},
  pdfauthor={Olga Alieva},
  hidelinks,
  pdfcreator={LaTeX via pandoc}}

\title{Liza}
\author{Olga Alieva}
\date{2023-07-21}

\begin{document}
\maketitle

\hypertarget{ux43fux435ux440ux435ux447ux438ux442ux44bux432ux430ux44f-ux431ux435ux434ux43dux443ux44e-ux43bux438ux437ux443-sentiment-analysis-ux434ux43bux44f-ux441ux435ux43dux442ux438ux43cux435ux43dux442ux430ux43bux44cux43dux43eux439-ux43fux440ux43eux437ux44b}{%
\section{Перечитывая ``Бедную Лизу'': Sentiment Analysis для
сентиментальной
прозы}\label{ux43fux435ux440ux435ux447ux438ux442ux44bux432ux430ux44f-ux431ux435ux434ux43dux443ux44e-ux43bux438ux437ux443-sentiment-analysis-ux434ux43bux44f-ux441ux435ux43dux442ux438ux43cux435ux43dux442ux430ux43bux44cux43dux43eux439-ux43fux440ux43eux437ux44b}}

Про ``Бедную Лизу'' Карамзина все помнят ровно столько, чтобы не
стремиться ее перечитывать: бедная крестьянская девушка влюбляется в
дворянина, который, разорившись, женится на богатой вдове. Лиза топится
в пруду. ``Слащавая'', по выражению Вайля и Гениса, повесть об
``опереточной'' крестьянке и ее ``добродетельной'' матушке. Чтобы читать
Карамзина сегодня, пишут они, нужно запастись ``эстетическим цинизмом,
позволяющим наслаждаться старомодным простодушием текста''.

Вместо эстетического цинизма мы, однако, запаслись инструментами
компьютерного анализа текста, чтобы с их помощью понять: все ли так
однозначно и простодушно в этой сентиментальной повести?

Анализ текста проводился на языке R с использованием IDE RStudio. Текст
``Лизы'' мы забрали с сайта Lib.ru.

Для лемматизации ипользовался пакет \texttt{udpipe} и модель, обученная
на корпусе \textbf{СинТагРус} (сокр. от англ.
\href{https://urlis.net/2yqwwngu}{Syntactically Tagged Russian text
corpus}, то есть «синтаксически аннотированный корпус русских
текстов»)\footnote{\url{https://universaldependencies.org/treebanks/ru_syntagrus/index.html}}.

Для анализа эмоциональной тональности мы выбрали лексикон
\href{https://www.labinform.ru/pub/rusentilex/index.htm}{RuSentiLex}
2016 г., доступный в пакете \texttt{rulexicon}. Он содержит около 15000
уникальных слов или фраз, среди которых оценочные слова, а также слова и
выражения, не передающие оценочное отношения автора, но имеющие
положительную или отрицательную ассоциацию (коннотацию). Возможные
значения переменной \texttt{sentiment}: neutral, positive, negative, а
также positive/negative. Для некоторых лемм мы внесли изменения вручную:
так, слова ``зеленый'', ``старый'', ``тучный'' у Карамзина не несут
негативной окраски, равно как и ``простой'', ``тихий'' и
``чувствительный''.

Выбрав из текста все существительные, мы c удивлением обнаружили, что
слова ``мать'', ``матушка'', ``старушка'' (42 раза) встречаются в тексте
в два раза чаще, чем слова ``Лиза'' или ``девушка'' (22 раза):

\begin{verbatim}
## # A tibble: 5 x 2
##   lemma        n
##   <chr>    <int>
## 1 мать        21
## 2 Лиза        16
## 3 матушка     11
## 4 старушка    10
## 5 девушка      6
\end{verbatim}

В повести о влюбленной девушке Карамзин гораздо чаще говорит о ее
матери, чем о ней самой! И здесь уместно напомнить, что самая знаменитая
фраза этой повести -- ``и крестьянки любить умеют!'' -- относится вовсе
не к Лизе, а к ее матери, трогательно ожидающей встречи со своим мужем
на том свете. С мужем, о котором ничего не известно, что звали его
заурядным именем Иван и жил он со своей женой в согласии.

Разделив весь текст ``Лизы'' на отрывки по 100 слов, мы решили
проверить, как меняется эмоциональная тональность произведения по мере
развития сюжета. В тексте чуть более 5000 слов, у нас получился 51
отрывок. Переменную \texttt{sentiment} мы перевели из категориального
(positive/negative) в числовой формат (-1 / 1); нейтральную лексику
удалили. Для некоторых неоднозначных слов (positive/negative) прямо
указали, как они окрашены у Карамзина. Это дало возможность сложить все
negative и positive в отрывке: там, где значение оказалось выше нуля,
преобладает положительная тональность, а там, где ниже -- отрицательная.
Результат можно видеть на графике:

\includegraphics{Liza_files/figure-latex/Тональность текста на оси повестовательного времени-1.pdf}

И снова текст Карамзина повел себя достаточно неожиданно.

Негативная тональность в начале повести (отрывок 5) не имеет ничего
общего с развитием сюжета: здесь такие слова, как ``неприятель'',
``опустошать'', ``свирепый''. Рассказчик, глядя на опустевший Симонов
монастырь, вспоминает о ``печальной истории'' Москвы. Таким образом, с
количественной точки зрения, самые мрачные фрагменты повести посвящены
не судьбе бедной девушки, а судьбе страны: Карамзин историк уже
переигрывает Карамзина-новелиста. Вот это место:

\begin{quote}
Иногда на вратах храма рассматриваю изображение чудес, в сем монастыре
случившихся, там рыбы падают с неба для насыщения жителей монастыря,
осажденного многочисленными врагами; тут образ богоматери обращает
неприятелей в бегство. Все сие обновляет в моей памяти историю нашего
отечества --- печальную историю тех времен, когда свирепые татары и
литовцы огнем и мечом опустошали окрестности российской столицы и когда
несчастная Москва, как беззащитная вдовица, от одного бога ожидала
помощи в лютых своих бедствиях.
\end{quote}

Называя Москву ``вдовицей'' Каразин тем самым связывает ее образ с
образом Лизиной матери, тоже бедной вдовы, за которую некому
заступиться. В отрывке 11 график отразил тревогу матери о судьбе дочери:
``У меня всегда сердце бывает не на своем месте, когда ты ходишь в
город; я всегда ставлю свечу перед образ и молю господа бога, чтобы он
сохранил тебя от всякой беды и напасти''. Здесь проявляется естественное
для сентименталиста противопоставление порочного города и мирной, чистой
природы.

Тревога оказывается ненапрасной. Отрывок 34 -- это падение Лизы:
``Грозно шумела буря, дождь лился из черных облаков --- казалось, что
натура сетовала о потерянной Лизиной невинности''. С точки зрения
эмоциональной окрашенности этот место даже более насыщено, чем эпизод
самоубийства Лизы: нравственная смерть Лизы случается раньше, чем
физическая, о которой Карамзин говорит, конечно, с грустью, но без
надрыва: ``Тут она бросилась в воду''.

Мрачную тональность последним страницам придает не столько известие о
гибели дочери, сколько скорбная участь ее матери: ``Лизина мать услышала
о страшной смерти дочери своей, и кровь ее от ужаса охладела --- глаза
навек закрылись''. Лиза, эта нежная и чувствительна натура, перед
смертью распоряжается насчет денег для матери, не задумываясь о том, что
и сам ее поступок, и этот дикий подарок убьют старушку. Ее
чувствительности не хватает на то, чтобы представить страдания другого
сердца.

В начале повести она говорит матери: ``Бог дал мне руки, чтобы
работать\ldots{} ты кормила меня своею грудью и ходила за мною, когда я
была ребенком; теперь пришла моя очередь ходить на тобою''. Об этих
словах она забывает, как только ее бросает любовник. Крестьянки умеют
любить -- да, но не все. И не всех.

Карамзин, конечно, жалеет девушку, как жалеет он и ветреного Эраста. И
жанр сентиментальной повести вынуждает его говорить прежде всего о
судьбе несчастной Лизы. Но, кажется, две вдовицы -- Москва и
старушка-мать -- вызывают не меньшее, а может быть и большее его
сочувствие.

\begin{center}\rule{0.5\linewidth}{0.5pt}\end{center}

\emph{Все сказанное следует понимать лишь как читательский опыт, не
претендующий на литературоведческую основательность.}

\end{document}
